\documentclass[aps,superscriptaddress,twocolumn,nopreprintnumbers,floatfix,groupedaddress]{revtex4-1}
\usepackage{amssymb}
\usepackage{amsmath}
\usepackage{graphicx}
\usepackage{dcolumn}
\usepackage{hyperref}
\usepackage{color,units}
\usepackage{lineno}
\usepackage{xspace}
\usepackage{mathtools}
\usepackage{physics}
\usepackage{acronym}
\usepackage{subfigure}

\newcommand{\bilby}{{\sc Bilby}\xspace}
\newcommand{\lal}{{\sc LAL}\xspace}
\newcommand{\lalsuite}{{\sc LALSuite}\xspace}
\newcommand{\lalsimulation}{{\sc LALSimulation}\xspace}
\newcommand{\sur}{{\sc NRHybSur3dq8}\xspace}
\newcommand{\Z}{\mathcal{Z}}
\newcommand{\M}{\mathcal{M}}
\renewcommand{\L}{\mathcal{L}}
\newcommand{\BF}{\mathcal{BF}}
\newcommand{\proposal}{proposal}
\newcommand{\target}{target}
\newcommand{\ep}[1]{\textcolor{red}{[EP: #1]}}
\newcommand{\et}[1]{\textcolor{blue}{[ET: #1]}}
\newcommand{\ct}[1]{\textcolor{green}{[CT: #1]}}

\newcommand{\nessai}{{\sc Nessai}\xspace}
\newcommand{\vitamin}{{\sc VItamin}\xspace}
\newcommand{\bilbypipe}{{\sc bilby\_pipe}\xspace}
\newcommand{\lalinference}{{\sc LALInference}\xspace}
\newcommand{\dynesty}{{\sc dynesty}\xspace}
\newcommand{\cpnest}{{\sc cpnest}\xspace}
\newcommand{\nflows}{{\sc nflows}\xspace}
\newcommand{\pytorch}{{\sc PyTorch}\xspace}
\newcommand{\corner}{{\sc corner}\xspace}
\newcommand{\matplotlib}{{\sc matplotlib}\xspace}
\newcommand{\seaborn}{{\sc seaborn}\xspace}
\newcommand{\numpy}{{\sc NumPy}\xspace}
\newcommand{\scipy}{{\sc SciPy}\xspace}
\newcommand{\pandas}{{\sc pandas}\xspace}
\newcommand{\python}{{\sc Python}\xspace}
\newcommand{\imrphenomp}{{\sc IMRPhenomPv2}\xspace}


\newcommand{\figwidth}{8.6cm}
\newcommand{\onehalffigwidth}{12.9cm}
\newcommand{\doublefigwidth}{17.2cm}
\newcommand{\montefigwidth}{11cm}

\begin{document}

\title{Explainable Deep-learning: Monte Carlo methods for Gravitational-Wave Inference}

\author{Project No:~628}
\affiliation{%
	SUPA, School of Physics and Astronomy \\
	University of Glasgow \\
	Glasgow G12 8QQ, United Kingdom}%

\date{\today}

\begin{abstract}
My 250 word abstract goes here...
\end{abstract}

\maketitle

\acrodef{GW}[GW]{Gravitational wave}
\acrodef{BBH}[BBH]{binary black hole}
\acrodef{EM}[EM]{electromagnetic}
\acrodef{CBC}[CBC]{compact binary coalescence}
\acrodef{BNS}[BNS]{binary neutron star}
\acrodef{NSBH}[NSBH]{neutron star black hole}
\acrodef{PSD}[PSD]{power spectral density}
\acrodef{ELBO}[ELBO]{evidence lower bound}
\acrodef{LIGO}[LIGO]{advanced Laser Interferometer Gravitational wave Observatory}
\acrodef{CVAE}[CVAE]{conditional variational autoencoder}
\acrodef{KL}[KL]{Kullback--Leibler}
\acrodef{GPU}[GPU]{graphics processing unit}
\acrodef{LVC}[LVC]{LIGO-Virgo Collaboration}
\acrodef{PP}[p-p]{probability-probability}
\acrodef{SNR}[SNR]{signal-to-noise ratio}

\section{Introduction}\label{intro}

\textbf{\textcolor{red}{Figs: Hunter's Vit Schematic}}

\textbf{\textcolor{green}{Tables: Compare inference speeds like in vit paper}}

Remember to signpost rest of paper at end of this section!

\subsection{Parameter Estimation}

\subsection{Deep-learning Approaches}

\subsection{\vitamin: User-friendly Inference}\label{vit}

\textbf{\textcolor{green}{Tables: Compare Gen Model abilities}}

Use gen pap to intro CVAE in context, CONTEXT IS KEY HERE


\begin{figure}
	\centering
	\includegraphics[width=\figwidth]{figs/network_setup.png}
	\caption{}
	\label{fig:vit_flow}
\end{figure}

%
%\subsection{Structure}
%
%\subsection{Training}
%
%\subsection{Results}

%\section{Theoretical Framework}\label{theory}

Need to mention metropolis hastings it seems!

Introduce equations directly to our specifics, we don't have space to intro them blind then again to specifics...

%\subsection{Monte Carlo Framework}\label{theory:monte}

%\subsection{SIR Framework}\label{theory:sir}

Do theory on normal IS and then say that SIR is an monte carlo approach/approx to normal IS then give equations for bot (talk about the NEW IMPROVED SIR method (link to Section \ref{future}))

%
%\subsection{Theoretical Framework}\label{monte:theory}

\section{Methodology}\label{methods}

Apply the intro/theory mateiral to our case, JUSTIFY scientific decisions like number of samples, batch size, npars!!

\subsection{Model Training}

\begin{figure}
	\centering
	\includegraphics[width=\figwidth]{figs/cost.pdf}
	\caption{}
	\label{fig:learning_contours}
\end{figure}

\begin{figure*}
	\subfigure{\includegraphics[width=\figwidth]{figs/vit_train_corner1.pdf}}
	\subfigure{\includegraphics[width=\figwidth]{figs/vit_train_corner2.pdf}}
	\caption{Probability-probability (P-P) plot showing the confidence interval versus the fraction of the events within that confidence interval for the posterior distributions obtained using our analysis \nessai for 128 simulated compact binary coalescence signals produced with \bilby and \bilbypipe. The 1-, 2- and 3-$\sigma$ confidence intervals are indicated by the shaded regions and $p$-values are shown for each of the parameters and the combined $p$-value is also shown.}
	\label{fig:vit_train_corner}
\end{figure*}

\textbf{\textcolor{red}{Figs: loss plot}}

\textbf{\textcolor{green}{Tables: training hypers in table}}

\textbf{\textcolor{red}{Figs: initial corner plot? (to talk about params and how posteriors aren't perfect)}}

Need this cornerplot here to talk about how it doesn't `get' the multimodal dists, which after resampling it does!

\subsection{Likelihood Estimates}

\textbf{\textcolor{red}{Figs: Monte flowchart}}

\begin{figure*}
	\centering
	\includegraphics[width=\montefigwidth]{figs/tikz_monte.pdf}
	\caption{}
	\label{fig:monte_flow}
\end{figure*}


\subsection{Likelihood Reweighting}


\section{Results}\label{results}

\subsection{Self-consistency}

\textbf{\textcolor{red}{Figs: Self consist corner plot}}

\begin{figure*}
	\subfigure{\includegraphics[width=\figwidth]{figs/self_consist1.pdf}}
	\subfigure{\includegraphics[width=\figwidth]{figs/self_consist2.pdf}}
	\caption{Probability-probability (P-P) plot showing the confidence interval versus the fraction of the events within that confidence interval for the posterior distributions obtained using our analysis \nessai for 128 simulated compact binary coalescence signals produced with \bilby and \bilbypipe. The 1-, 2- and 3-$\sigma$ confidence intervals are indicated by the shaded regions and $p$-values are shown for each of the parameters and the combined $p$-value is also shown.}
	\label{fig:sel_consist}
\end{figure*}


\subsection{Reproducibility}

Talk about how 'binning' is preventing proper error profile acorss the likelihood range, (not present in the \dynesty case)

\textbf{\textcolor{red}{Figs: sigma gaussians for different z batch}}

\begin{figure}
	\centering
	\includegraphics[width=\figwidth]{figs/hists_rect.pdf}
	\caption{}
	\label{fig:hists}
\end{figure}



\textbf{\textcolor{red}{Figs: scatter vit vit}}

\textbf{\textcolor{red}{Figs: scatter vit dynesty}}


\begin{figure*}
	\subfigure[]{\includegraphics[width=\figwidth]{figs/vvscatter.pdf}}
	\subfigure[]{\includegraphics[width=\figwidth]{figs/bvscatter.pdf}}
	\caption{Probability-probability (P-P) plot showing the confidence interval versus the fraction of the events within that confidence interval for the posterior distributions obtained using our analysis \nessai for 128 simulated compact binary coalescence signals produced with \bilby and \bilbypipe. The 1-, 2- and 3-$\sigma$ confidence intervals are indicated by the shaded regions and $p$-values are shown for each of the parameters and the combined $p$-value is also shown.}
	\label{fig:scatter}
\end{figure*}

\subsection{Importance Resampling}


\textbf{\textcolor{red}{Figs: Final corner plot (big)}}

\begin{figure*}[h]
	\centering
	\includegraphics[width=\doublefigwidth]{figs/resample_corner.pdf}
	\caption{}
	\label{fig:final_corner}
\end{figure*}

%\section{Future Work}\label{future}


\section{Conclusions}\label{conc}

This is section has to encapsulate everything we did so that after the abstract a reader can go here and see if they want to buy the paper or not!

As we find ourself in a proof-of-concept mode, there is justification of a section dedicated to the next steps leading towards production of this code.


\section*{Acknowledgements}

Thanks to Chris and Hunter and Michael and Daniel.

Paragraph on the software used \bilby\cite{bilby} 
%\clearpage
\bibliography{refs}




\end{document}